\input{preamble}

\begin{document}

% indentation
\setlength\baselineskip{13.4pt} 
\setlength\parskip{6pt}
\setlength\parindent{0pt}
\abovedisplayskip=6pt % displaymath
\belowdisplayskip=6pt % displaymath
\raggedright          % forced linebreaks

\section{Лабораторная №1}

\subsection{1 Текст задания}

Написать программу на языке Java, выполняющую указанные в варианте действия.

{\ital Требования к программе:}
% ---
\begin{list*}[][\#]
\item Программа должна корректно запускаться, выполняться и выдавать результат. Программа не должна выдавать ошибки. Программа должна быть работоспособной именно во время проверки, то, что она работала 5 минут назад, дома или в параллельной вселенной оправданием не является.
\item Выражение должно вычисляться в соответствии с правилами вычисления математических выражений {\ital\color{desc} (должен соблюдаться порядок выполнения действий и т.д.)}.
\item Программа должна использовать математические функции из стандартной библиотеки Java.
\item Вычисление очередного элемента двумерного массива должно быть реализовано в виде отдельного статического метода.
\item Результат вычисления выражения должен быть выведен в стандартный поток вывода в виде матрицы с элементами в указанном в варианте формате. Вывод матрицы реализовать в виде отдельного статического метода.
\item Программа должна быть упакована в исполняемый jar-архив.
\item Выполнение программы необходимо продемонстрировать на сервере helios.
\end{list*}
% ---
{\ital Примечания:}
% ---
\begin{list*}[][\#]
\item В случае, если в варианте будут предложены одинаковые имена массивов, для одного из них к имени добавить "1".
\item Если в результате вычислений иногда получается NaN --- возможно так и должно быть.
\end{list*}

\newpage
{\ital Вариант 30449:}
% ---
\begin{list*}[][\#]
\item Создать одномерный массив $w$ типа long. Заполнить его нечётными числами от $7$ до $19$ включительно в порядке убывания.
\item Создать одномерный массив $x$ типа double. Заполнить его $11$-ю случайными числами в диапазоне от $-9.0$ до $4.0$.
\item Создать двумерный массив $f$ размером $7\times 11$. Вычислить его элементы по следующей формуле {\ital (где $x=x[j]$)}:
% ---
\begin{list*}[2]
\item если $w[i] = 11$, то:
% ---
$$f[i][j]=\cos(\sin(\sqrt[3]{x}))$$
\item если $w[i]\in\{13,15,19\}$, то:
% ---
$$f[i][j]=\left(\frac{4}{\left(2\cdot\arcsin\left(\frac{x-2.5}{13}\right)\right)^{\left(\frac{x}{x-3}\right)^2}}\right)^3$$
\item для остальных значений w[i]:
% ---
$$f[i][j]=2\cdot\frac{1}{2}\cdot\left(\frac{\frac{2}{3}-\sqrt[3]{x}}{\pi}\right)^3$$
\end{list*}
\item Напечатать полученный в результате массив в формате с~тремя знаками после запятой.
\end{list*}

\subsection{2 Исходный код программы}

\href{https://github.com/MrDvD/itmo_labs/blob/master/lab1/lab1.java}{Ссылка} на исходный код программы.

\subsection{3 Результат работы программы}

\href{https://github.com/MrDvD/itmo_labs/blob/master/lab1/file.out}{Ссылка} на результат работы программы с примером входных данных.

\newpage
\subsection{4 Выводы по работе}

В ходе выполнения лабораторной работы я познакомился с синтаксисом языка программирования Java и отдельно изучил некоторые функции класса Math. Также я узнал об особенностях логики операции деления при работе с разными типами данных {\ital (long, double)}. Ещё мне приходилось изучать причины возникавших ошибок при компиляции кода и самостоятельно их исправлять. Наконец, для демонстрации своего решения на сервере helios я вспомнил, как работать с утилитами {\ital ssh}, {\ital scp} и {\ital vim}.

\end{document}
